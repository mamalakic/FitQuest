\section{Όροι \& αντικείμενα}

\begin{itemize}
    \item \textbf{Player (Παίκτης)}: Είναι απλός χρήστης του συστήματος. Ο κάθε παίκτης έχει φίλους, \underline{αντικέιμενα}, \underline{προγράμματα εκγύμνασης} και μπορεί να ανήκει σε \underline{ομάδα}.
    \item \textbf{Item (Αντικείμενο)}: Τα αντικείμενα είναι οντότητες που μπορούν να αγοραστούν ή αποκτηθούν από τον παίκτη και τον βοηθούν στην πρόοδο του.
    \item \textbf{Shop (Κατάστημα)}: Το κατάστημα είναι η οντότητα που περιέχει τα αντικείμενα που μπορεί να αγοράσει/πουλήσει ο παίκτης.
    \item \textbf{Training\_Programme (Πρόγραμμα γυμναστικής)}: Μία συλλογή ασκήσεων που έχει ο κάθε παίκτης και χρησιμοποιείται κατά τις \underline{μάχες}. Μία άσκηση μπορεί να περιέχεται περισσότερες από μία φορές σε ένα πρόγραμμα (πολλά σετ της άσκησης).
    \item \textbf{Exercise (Άσκηση)}: Κάθε ασκήση αντιπροσωπεύει μία ενέργεια που πρέπει να εκτελέσει ο παίκτης για να νικήσει έναν \underline{εχθρό}.
    \item \textbf{Team (Ομάδα)}: Αποτελείται από πολλούς παίκτες, οι οποίοι συνεργάζονται στις \underline{ομαδικές μάχες}.
    \item \textbf{Battle (Μάχη)}: Μια λειτουργία κατά την οποία ο χρήστης εκγυμνάζεται ώστε να κερδίσει έναν εχθρό του συστήματος (όχι παίκτη).
    \item \textbf{Team\_Battle (Ομαδική Μάχη)}: Όπως σε μία απλή μάχη, μόνο που σε αυτήν συμμετέχουν πολλοί παίκτες της ίδιας ομάδας.
    \item \textbf{Opponent (Εχθρός)}: Ο εχθρός είναι ένας χαρακτήρας του παιχνιδιού, ο οποίος αντιπροσωπεύει τον αντίπαλο του παίκτη σε μια μάχη. Ο χρήστης πρέπει να τον νικήσει εκτελώντας ασκήσεις με την υποβοήθεια της κάμερας.
    \item \textbf{Map (Χάρτης)}: Ο χάρτης αντιπροσωπεύει την πρόοδο του παίκτη στο παιχνίδι. Περιέχει \underline{περιοχές}.
    \item \textbf{Area (Περιοχή)}: Κάθε περιοχή αντιπροσωπεύει πιο συγκεκριμένα την πρόοδο του παίκτη. Μπορεί να έχει διάφορες επιρρόες στις ασκήσεις του παίκτη και περιέχει πολλαπλές μάχες.
\end{itemize}
