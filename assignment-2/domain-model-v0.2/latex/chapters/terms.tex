\section{Όροι \& αντικείμενα}

\begin{itemize}
    \item \textbf{Player (Παίκτης)}: Είναι απλός χρήστης του συστήματος. Ο κάθε παίκτης έχει \underline{\textcolor{red}{μια λίστα φίλων}}, \underline{\textcolor{red}{σακίδιο}} \underline{\textcolor{red}{αντικειμένων}}, \underline{\textcolor{red}{πρόγραμμα} εκγύμνασης}, \underline{\textcolor{red}{συνεδρία}} και μπορεί να ανήκει σε \underline{ομάδα}.
    
    \item \textcolor{red}{\textbf{Friend\_List {Λίστα Φίλων}}: Περιέχει τους φίλους ενός παίκτης.}
    
    \item \textcolor{red}{\textbf{Inventory (Σακίδιο Αντικειμένων)}: Το σακίδιο αντικειμένων είναι η οντότητα που περιέχει τα \underline{αντικείμενα} που έχει αποκτήσει ο παίκτης.}
    
    \item \textcolor{red}{\textbf{Session (Συνεδρία)}: Το αντικείμενο αντιπροσωπεύει την τρέχουσα συνεδρία του παίκτη. Περιέχει πληροφορίες σχετικά με την συνδεσιμότητα του παίκτη με το σύστημα/διαδίκτυο.}

    \item \textbf{Item (Αντικείμενο)}: Τα αντικείμενα είναι οντότητες που μπορούν να αγοραστούν από το \underline{\textcolor{red}{κατάστημα}} ή αποκτηθούν \textcolor{red}{από διάφορα επιτέυγματα του παίκτη} και τον βοηθούν στην πρόοδο του.
    
    \item \textbf{Shop (Κατάστημα)}: Το κατάστημα είναι η οντότητα που περιέχει τα αντικείμενα που μπορεί να αγοράσει/πουλήσει ο παίκτης.
    
    \item \textbf{Training\_Programme (Πρόγραμμα \textcolor{red}{εκγύμνασης})}: Μία συλλογή ασκήσεων που έχει ο κάθε παίκτης και χρησιμοποιείται κατά τις \underline{μάχες}. Μία άσκηση μπορεί να περιέχεται περισσότερες από μία φορές σε ένα πρόγραμμα (πολλά σετ της άσκησης).
    
    \item \textbf{Exercise (Άσκηση)}: Κάθε ασκήση αντιπροσωπεύει μία ενέργεια που πρέπει να εκτελέσει ο παίκτης για να νικήσει έναν \underline{εχθρό}.
    
    \item \textbf{Team (Ομάδα)}: Αποτελείται από πολλούς παίκτες, οι οποίοι συνεργάζονται στις \underline{ομαδικές μάχες}.
    
    \item \textcolor{red}{\textbf{Team Form (Φόρμα Δημιουργίας Ομάδας)}: Χρησιμοποιείται κατά την δημιουργία μιας ομάδας.}
    
    \item \textbf{Battle (Μάχη)}: Μια λειτουργία κατά την οποία ο χρήστης εκγυμνάζεται ώστε να κερδίσει έναν εχθρό του συστήματος (όχι παίκτη).
    
    \item \textbf{Team\_Battle (Ομαδική Μάχη)}: Όπως σε μία απλή μάχη, μόνο που σε αυτήν συμμετέχουν πολλοί παίκτες της ίδιας ομάδας.
    
    \item \textbf{\textcolor{red}{Enemy} (Εχθρός)}: Ο εχθρός είναι ένας χαρακτήρας του παιχνιδιού, ο οποίος αντιπροσωπεύει τον αντίπαλο του παίκτη σε μια μάχη. Ο χρήστης πρέπει να τον νικήσει εκτελώντας ασκήσεις με την υποβοήθεια της κάμερας.
    
    \item \textbf{Map (Χάρτης)}: Ο χάρτης αντιπροσωπεύει την πρόοδο του παίκτη στο παιχνίδι. Περιέχει \underline{περιοχές}.
    
    \item \textbf{\textcolor{red}{Level} (Περιοχή)}: Κάθε περιοχή αντιπροσωπεύει πιο συγκεκριμένα την πρόοδο του παίκτη. Μπορεί να έχει διάφορες επιρρόες στις ασκήσεις του παίκτη και περιέχει πολλαπλές μάχες.
    
    \item \textcolor{red}{\textbf{Database\_Communcation (Επικοινωνία με την Βάση Δεδομένων)}: Χρησιμοποιείται για την αποθήκευση και ανάκτηση δεδομένων από την βάση δεδομένων.}
    
    \item \textcolor{red}{\textbf{Message\_Window (Παράθυρο Μηνύματος)}: Μαζί με τις υποκλάσσεις, αποτελούν μερικές οντότητες που χρειάζονται για την μετάβαση πληροφορίας από και προς τον παίκτη}
\end{itemize}
