\section{Πιθανοί χειριστές}

Χρήστης/Παίκτης: Ο χειριστής της εφαρμογής απο τον προσωπικό υπολογιστή του. Αλλιώς γνωστός ως "πελάτης".
\newline
Σύστημα/Παιχνίδι: Η εγκατεστημένη εφαρμογή που ανταποκρίνεται στις ενέργειες του χρήστη. Επικοινωνεί με τους διακομιστές. Στο μοντέλο περιπτώσεων χρήσης αντιστοιχίζεται μόνο στα use case που εκτελεί κάτι πέρα απο την εναλλαγή οθονών κλπ. για λόγους ευκρίνειας.
\newline
Βάση Δεδομένων: Η ομάδα σκληρών δίσκων της εφαρμογής που απαρτίζουν τον χώρο αποθήκευσης των δεδομένων των χρηστών. 

\section{Περιπτώσεις χρήσης που θα υλοποιηθούν}
Σε επαρκή λειτουργικότητα θα υλοποιηθούν:
\begin{enumerate}
\item \hyperref[sec:profile]{Εξατομίκευση προγράμματος}
\item \hyperref[sec:buy]{Αγορά αντικειμένου}
\item \hyperref[sec:clanbattle]{Μάχη ομάδας}
\item \hyperref[sec:solo]{(Σόλο) μάχη}
\item \hyperref[sec:map]{Χάρτης περιπέτειας}
\item \hyperref[sec:backpack]{Σακίδιο αντικειμένων}
\item \hyperref[sec:createclan]{Δημιουργία ομάδας}
\item \hyperref[sec:friendslist]{Λίστα φίλων}
\end{enumerate}

\newpage
\section{Συνολικό μοντέλο περιπτώσεων χρήσης}

\begin{figure}[htbp]
  
  \centering
    \includegraphics[width=0.9\textwidth]{graphics/uml.png}
    \caption{Rendered using UMLet}
\end{figure}

\newpage
\section{Κείμενα περιπτώσεων χρήσης}
\subsection{Εξατομίκευση προγράμματος}
\label{sec:profile}
\begin{enumerate}
    \item Ο χρήστης επιχειρεί να προπονηθεί.
    \item Το σύστημα ελέγχει εάν ο εγγεγραμμένος χρήστης έχει διαμορφώσει το προφίλ του και διαπιστώνει πως αυτό δεν έχει γίνει.
    \item Το σύστημα προωθεί τον χρήστη να διαμορφώσει το προφίλ του.
    \item Ο χρήστης συμπληρώνει διάφορα πεδία και παραμέτρους για να αξιολογηθεί το επίπεδο ενασχόλησης του με την γυμναστική και η φυσική του υγεία.
    \item Το σύστημα ανακτά τις τιμές που εισήγαγε ο χρήστης και μέσα από αλγόριθμο προτείνει μερικά προγράμματα γυμναστικής που δεν έχουν επιλεχθεί ήδη από τον χρήστη.
    \item Ο χρήστης επιλέγει να ακολουθήσει ένα από αυτά.
    \item Το σύστημα αποθηκεύει την επιλογή του χρήστη και στην επόμενη μάχη θα εμφανίσει τις ασκήσεις του προγράμματος.
\end{enumerate}

Εναλλακτική ροή 1:
\begin{enumerate}[label=2.\alph*.,ref=2.\alph*]


    \item Το σύστημα διαπιστώνει πως ο χρήστης έχει διαμορφώσει ήδη το προφίλ του.
    \item Το σύστημα ανακτά τις τιμές του προφίλ του χρήστη και μέσα από αλγόριθμο προτείνει μερικά προγράμματα γυμναστικής.
    \item Ο χρήστης επιλέγει να ακολουθήσει ένα από αυτά.
    \item Το σύστημα αποθηκεύει την επιλογή του χρήστη και στην επόμενη μάχη θα εμφανίσει τις ασκήσεις του προγράμματος.
\end{enumerate}

Εναλλακτική ροή 2:
\begin{enumerate}[label=5.\alph*.,ref=2.\alph*]
    \item Ο χρήστης επιλέγει να μην ακολουθήσει ένα από αυτά.
    \item Το σύστημα του εμφανίζει τις επιλογές να ακολουθήσει ένα πρόγραμμα που ακολούθησε στο παρελθόν ή να δημιουργήσει δικό του.
    \item Ο χρήστης επιλέγει να δημιουργήσει δικό του πρόγραμμα.
    \item Το σύστημα του εμφανίζει οθόνη με τις ασκήσεις που μπορεί να επιλέξει.
    \item Ο χρήστης επιλέγει τις ασκήσεις και ολοκληρώνει την διαμόρφωση προγράμματός του.
\end{enumerate}

\newpage
\subsection{Αγορά αντικειμένου}
\label{sec:buy}
\begin{enumerate}
    \item Ο χρήστης εισέρχεται στην οθόνη της αγοράς
    \item Το σύστημα μεταφέρει τον χρήστη στην αγορά, ανακτά τα πιο δημοφιλή αντικείμενα και τα βάζει πρώτα στην λίστα.
    \item Ο χρήστης εφαρμόζει συγκεκριμένα φίλτρα για να βρει τα αντικείμενα που τον ενδιαφέρουν.
    \item Το σύστημα βρίσκει τα αντικείμενα με βάση την επιλογή φίλτρων και τα εμφανίζει στον χρήστη.
    \item Ο χρήστης επιλέγει το αντικείμενο που θέλει να αγοράσει.
    \item Το σύστημα ελέγχει εάν ο χρήστης έχει αρκετά νομίσματα παιχνιδιού ώστε να αγοράσει το αντικείμενο. Αυτό ισχύει άρα αφαιρεί το ποσό από το υπόλοιπο του χρήστη, επιβεβαιώνει την συναλλαγή και προσθέτει το αντικείμενο στον "σάκο αντικειμένων" του χρήστη.
    \item Το σύστημα επιστρέφει τον χρήστη στην οθόνη της αγοράς.
\end{enumerate}

Εναλλακτική ροή 1:
\begin{enumerate}[label=6.\alph*.,ref=6.\alph*]
    \item Το σύστημα διαπιστώνει ένα σφάλμα κατά την επαλήθευση της συναλλαγής. Αναιρεί κάθε προηγούμενη ενέργεια και ενημερώνει τον χρήστη. 
    \item Το σύστημα δίνει στον χρήστη την επιλογή να ξαναπροσπαθήσει την συναλλαγή ή να την ακυρώσει.
    \item Ο χρήστης ξαναπροσπαθεί να αγοράσει το αντικείμενο.
    \item Το σύστημα επαναλαμβάνει τα παραπάνω μέχρι να επαληθευτεί η συναλλαγή.
\end{enumerate}

\newpage
\subsection{Μάχη ομάδας}
\label{sec:clanbattle}
\begin{enumerate}
\item Ο χρήστης επιλέγει να πολεμήσει μαζί με άλλους παίκτες.
\item Το σύστημα ελέγχει εάν ο χρήστης έχει ενεργοποιημένο ίντερνετ και εάν είναι σε κάποια ομάδα. Διαπιστώνοντας πώς ισχύουν και τα δύο του εμφανίζει την επιλογή.
\item Ο χρήστης επιλέγει να προχωρήσει.
\item Το σύστημα ανακτά την πρόοδο της ομάδας και την εμφανίζει στον χρήστη. Εμφανίζει επίσης επιλογές αντιπάλων.
\item Ο χρήστης επιλέγει την μάχη που θέλει να πολεμήσει και ξεκινάει η μάχη ομάδας. Στέλνεται ενημέρωση σε κάθε μέλος της ομάδας. Ο τρόπος διεξαγωγής μάχης περιγράφεται στην περίπτωση χρήσης "(Σόλο) μάχη".
\item Κάθε μέλος της ομάδας μπορεί να πάρει μέρος στην μάχη ομάδας όσες φορές θέλει μέχρι την λήξη του χρονικού ορίου.
\item Κάθε μέλος κατά την είσοδό του στη μάχη ομάδας επιλέγει ποιές ασκήσεις (πρόγραμμα γυμναστικής) θα εκτελέσει.
\item Αν η ομάδα νικήσει τον αντίπαλο στον προαναφερόμενο χρόνο, η μάχη ομάδας θεωρείται νικημένη και το σύστημα απονέμει αμοιβές/λάφυρα.
\end{enumerate}

Εναλλακτική ροή 1:
\begin{enumerate}[label=2.\alph*.,ref=2.\alph*]
\item Το σύστημα διαπιστώνει πως δεν έχει ενεργοποιημένο ίντερνετ ο χρήστης και σταματάει την είσοδό του. Τον ενημερώνει με κατάλληλο μήνυμα.
\item Ο χρήστης ενεργοποιεί το ίντερνετ.
\item Το σύστημα διαπιστώνει πως πλέον έχει ίντερνετ και συνεχίζει στην επόμενη οθόνη.
\end{enumerate}

Εναλλακτική ροή 2:
\begin{enumerate}[label=2.\alph*.,ref=2.\alph*]
\item Το σύστημα διαπιστώνει πως ο χρήστης δεν ανήκει σε κάποια ομάδα οπότε σταματάει την είσοδό του. Τον ενημερώνει με κατάλληλο μήνυμα. Τον προτρέπει να κάνει εύρεση ομάδας.
\item Ο χρήστης κάνει εύρεση ομάδας από τις επιλογές της εφαρμογής.
\item Ο χρήστης μπαίνει σε μια ομάδα και επιστρέφει στην οθόνη επιλογής μάχης.
\item Το σύστημα ξανακάνει τους ελέγχους και εφόσον διαπιστώσει πως είναι πλέον σε ομάδα εμφανίζει την επόμενη οθόνη.
\end{enumerate}

Εναλλακτική ροή 3:
\begin{enumerate}[label=6.\alph*.,ref=6.\alph*]
\item Ένα μέλος της ομάδας προσπαθεί να συμμετάσχει.
\item Το σύστημα διαπιστώνει πως το μέλος της ομάδας που προσπαθεί να συμμετάσχει έχει προσπαθήσει ήδη πολλές φορές. Ρωτά αν ο χρήστης είναι κουρασμένος ή όχι, ενημερώνοντάς τον για τον κίνδυνο τραυματισμού.
\item Ο χρήστης διαβεβαιώνει ότι είναι εντάξει και ότι αναγνωρίζει τους κινδύνους της υπερκόπωσης και του τραυματισμού.
\item Το σύστημα ξεκινάει την μάχη ομάδας για τον χρήστη και εμφανίζει την οθόνη μάχης ομάδας.
\end{enumerate}

\newpage
\subsection{(Σόλο) μάχη}
\label{sec:solo}
\begin{enumerate}
    \item Ο χρήστης επιλέγει τις ασκήσεις που θα εκτελέσει μόνος του ή επιλέγει το προτεινόμενο πρόγραμμα γυμναστικής και ξεκινάει τη μάχη. %ALTERNATIVE SUGGEST
    \item Το σύστημα υπολογίζει τα στατιστικά του αντιπάλου με βάση την σωματική δύναμη του χρήστη, την δύναμη των αντικειμένων του και την πρόοδό του στον χάρτη, που αντλεί απο την βάση δεδομένων.
    \item Το σύστημα εμφανίζει την οθόνη μάχης με τον αντίπαλο και τους πόντους ζωής του.
    \item Ο χρήστης προκαλεί ζημιά στον αντίπαλο με κάθε επανάληψη (ασκήσεων.)
    \item Όταν η ζωή του αντιπάλου φτάσει στο 0, το σύστημα συγχαίρει τον χρήστη και τον ανταμείβει [πόντους εμπειρίας, αντικείμενα, νομίσματα]. Ενημερώνει το "προφίλ" του χρήστη στη βάση δεδομένων με τις ανταμοιβές του και καταγράφει το αρχείο της μάχης.
    \item Ο χρήστης επιστρέφει στην οθόνη του χάρτη.
\end{enumerate}

Εναλλακτική ροή 1:
\begin{enumerate}[label=5.\alph*.,ref=2.\alph*]
    \item Ο χρήστης δεν καταφέρνει να κερδίσει τον αντίπαλο (π.χ. κούραση, ακυρώνει τη μάχη...)
    \item Το σύστημα σταματάει αμέσως την μάχη. Ενημερώνει τον χρήστη ότι δεν θα πάρει ανταμοιβή.
    \item Ο χρήστης επιστρέφει στην οθόνη του χάρτη.
\end{enumerate}

Εναλλακτική ροή 2:
\begin{enumerate}[label=2.\alph*.,ref=2.\alph*]
    \item Το σύστημα ανακτά το ιστορικό του χρήστη και παρατηρείται ότι έχει καιρό να εκτελέσει κάποιες άλλες ασκήσεις.
    \item Το σύστημα προτείνει στον χρήστη να εκτελέσει άλλη άσκηση απο αυτήν που επέλεξε.
    \item Ο χρήστης συμφωνεί στην αλλαγή ή όχι της άσκησης στην προτεινόμενη. Η μάχη ξεκινάει.
\end{enumerate}

Εναλλακτική ροή 2:
\begin{enumerate}[label=2.\alph*.,ref=2.\alph*]
    \item Το σύστημα ανακτά το ιστορικό του χρήστη και υπολογίζεται ότι είναι προτιμότερο να μην εκτελέσει ασκήσεις (rest day, ή διάλειμμα αορίστου χρόνου).
    \item Το σύστημα προτείνει στον χρήστη να ξεκουραστεί, ενημερώνοντάς τον για τους κινδύνους τραυματισμού, ή να κάνει ένα πρόγραμμα γυμναστικής μικρότερης έντασης.
    \item Ο χρήστης επιλέγει αυτό που θέλει, διαβεβαιώνοντας ότι γνωρίζει τους κινδύνους.
\end{enumerate}

\newpage
\subsection{Χάρτης περιπέτειας}
\label{sec:map}
\begin{enumerate}
    \item Ο χρήστης εισέρχεται στην οθόνη της περιπέτειας.
    \item Το σύστημα εμφανίζει το χάρτη στο χρήστη και με βάση την αποθηκευμένη πρόοδό του του δείχνει τα κλειδωμένα επίπεδα, το επίπεδο που βρίσκεται και τα ήδη εκπληρωμένα επίπεδα. 
    \item Ο χρήστης επιλέγει επίπεδο.
    \item Το σύστημα φορτώνει την οθόνη μάχης και ζητάει από το χρήστη να στήσει την κάμερά του.
    \item Ο χρήστης στήνει την κάμερά του.
    \item Το σύστημα ανιχνεύει την κάμερα του χρήστη. 
    \item Το σύστημα, χρησιμοποιώντας παραμέτρους του ζωντανού βίντεο, ελέγχει την γωνία της κάμερας και επαληθεύει ότι το μέτρημα των επαναλήψεων μπορεί να γίνει κανονικά.
    \item Το σύστημα φορτώνει τις μπάρες ζωής των συμμετεχόντων και η μάχη ξεκινάει.
\end{enumerate}


Εναλλακτική ροή 1: 
\begin{enumerate}[label=3.\alph*.,ref=3.\alph*]
\item Ο χρήστης επιλέγει κλειδωμένο επίπεδο.
\item Το σύστημα ενημερώνει το χρήστη πως για να ξεκλειδώσει αυτό το επίπεδο πρέπει πρώτα να ολοκληρώσει τα προηγούμενα.
\item Το σύστημα οδηγεί τον χρήστη στο κομμάτι του χάρτη με το τελευταίο ξεκλειδωμένο επίπεδο.
\item Ο χρήστης επιλέγει ξανά επίπεδο.
\end{enumerate}


Εναλλακτική ροή 2:
\begin{enumerate}[label=5.\alph*.,ref=5.\alph*]
\item Ο χρήστης δεν έχει κάμερα.
\item Το σύστημα δεν ανιχνεύει την κάμερα του χρήστη.
\item Το σύστημα ενημερώνει το χρήστη πως δεν μπορεί να εμπλακεί σε μάχη χωρίς τη χρήση κάμερας.
\item Το σύστημα επιστρέφει το χρήστη πίσω στο χάρτη και δεν το αφήνει να μπει σε μάχη μέχρι να ανιχνεύσει κάμερα.
\end{enumerate}

Εναλλακτική ροή 3:
\begin{enumerate}[label=7.\alph*.,ref=7.\alph*]
\item Το σύστημα δεν μπορεί να εξακριβώσει την οπτική γωνία, με συνέπεια να μην μπορει να εξασφαλίσει σωστό μέτρημα των επαναλήψεων.
\item Το σύστημα ενημερώνει τον χρήστη για αυτό και τον συμβουλεύει να αλλάξει την θέση της κάμερας και να ξαναπροσπαθήσει. 
\item Ο χρήστης συνεχίζει τις αλλαγές μέχρι το σύστημα να μπορεί να εξακριβώσει τα απαραίτητα για την διεξαγωγή της μάχης.
\end{enumerate}

\newpage
\subsection{Σακίδιο αντικειμένων}
\label{sec:backpack}
\begin{enumerate}
\item Ο χρήστης ανοίγει το σακίδιο με τα αντικείμενά του.
\item Το σύστημα φορτώνει μια λίστα με τα αντικείμενα του χρήστη.
\item Ο χρήστης επιλέγει ένα αντικείμενο.
\item Το σύστημα δείχνει τις ιδιότητες/χρησιμότητες του αντικειμένου, το κόστος πώλησης και τον αριθμό των αντικειμένων αυτού του τύπου στην κατοχή του χρήστη.
\item Ο χρήστης επιλέγει να χρησιμοποιήσει ένα αντικείμενο.
\item Το σύστημα προσθέτει τις ιδιότητες του αντικειμένου στα στατιστικά του χρήστη και το αφαιρεί από το σακίδιο του χρήστη.
\end{enumerate}

Εναλλακτική ροή 1:
\begin{enumerate}[label=3.\alph*.,ref=3.\alph*]
\item Ο χρήστης επιλέγει ένα αντικείμενο που αποτελεί κομμάτι εξοπλισμού.
\item Το σύστημα δείχνει τα στατιστικά του αντικείμενου και το συγκρίνει με το ήδη φορεμένο κομμάτι εξοπλισμού (αν υπάρχει).
\item O χρήσης επιλέγει να φορέσει το νέο αντικείμενο.
\item Το σύστημα ανανεώνει τα στατιστικά του χρήστη.
\item Το σύστημα αφαιρεί το αντικείμενο από το σακίδιο και προσθέτει στη θέση του το πρώην φορεμένο (αντικατάσταση).
\end{enumerate}

Εναλλακτική ροή 2:
\begin{enumerate}[label=5.\alph*.,ref=5.\alph*]
\item Ο χρήστης επιλέγει να πουλήσει ένα αντικείμενο.
\item Το σύστημα μεταφέρει το χρήστη στην οθόνη της αγοράς.
\item Το σύστημα στέλνει επιβεβαιωτικό μήνυμα για την πώληση του αντικειμένου στο χρήστη.
\item Ο χρήστης επιβεβαιώνει τη συναλλαγή.
\item Το σύστημα προσθέτει τα νομίσματα στο λογαριασμό του χρήστη και αφαιρεί το αντικείμενο από το σακίδιο του χρήστη.
\end{enumerate}

\newpage
\subsection{Δημιουργία ομάδας}
\label{sec:createclan}
\begin{enumerate}
\item Ο χρήστης εισέρχεται στην οθόνη του στρατοπέδου ομάδας.
\item Το σύστημα τον ενημερώνει ότι δεν είναι σε ομάδα
\item Το σύστημα προτρέπει τον χρήστη να δημιουργήσει μια δική του ομάδα.
\item Ο χρήστης επιλέγει να δημιουργήσει μια ομάδα.
\item Μεταβαίνει στην οθόνη δημιουργίας ομάδας.
\item Το σύστημα του ζητάει να συμπληρώσει πεδία για τα χαρακτηριστικά της ομάδας και των ρυθμίσεών της.
\item Ο χρήστης δημιουργεί επιτυχώς την ομάδα του.
\item Το σύστημα εμφανίζει την λίστα φίλων του και του προτείνει να προσκαλέσει φίλους του στην ομάδα.
\item Ο χρήστης επιλέγει φίλους του.
\item Το σύστημα αποστέλλει ειδοποίηση στους επιλεγμένους φίλους με μια πρόσκληση στην ομάδα.
\item Επιστρέφει στην οθόνη του στρατοπέδου ομάδας.
\end{enumerate}

Εναλλακτική ροή 1:
\begin{enumerate}[label=6.\alph*.,ref=6.\alph*]
\item Διαπιστώνεται ότι μερικά πεδία δεν είναι έγκυρα. Σε περίπτωση στοιχείων όπως όνομα ομάδας, το σύστημα ελέγχει την βάση δεδομένων σε περίπτωση διπλοτυπίας.
\item Η ομάδα δεν δημιουργίεται και επισημαίνονται τα αντίστοιχα πεδία προς διόρθωση. 
\item Ο χρήστης τα διορθώνει και επιβεβαίωνει την δημιουργία ομάδας.
\item Η ομάδα δημιουργείται και η ροή συνεχίζεται κανονικά.
\end{enumerate}

Εναλλακτική ροη 2:
\begin{enumerate}[label=3.\alph*.,ref=3.\alph*]
\item Το σύστημα προτείνει στον χρήστη να μπει σε μια ήδη υπάρχουσα ομάδα. Του εμφανίζει μια λίστα ομάδων φίλων του.
\item Ο χρήστης επιλέγει μια ομάδα απο τη λίστα.
\item Το σύστημα τον εντάσσει στην ομάδα και αποστέλλει ειδοποίηση στα υπόλοιπα μέλη για την είσοδο του χρήστη.
\item Ο χρήστης μεταβαίνει στην οθόνη του στρατοπέδου ομάδας.
\end{enumerate}

\newpage
\subsection{Στρατόπεδο ομάδας}
\label{sec:clanbase}
\begin{enumerate}
\item Ο χρήστης εισέρχεται στην οθόνη του στρατοπέδου ομάδας.
\item Το σύστημα ανακτά τη λίστα μελών της ομάδας και τα παρουσιάζει στην οθόνη. Φορτώνουν επίσης ενέργεις που μπορεί να κάνει ο χρήστης (συνομιλία ομάδας, πρόσκληση σε μάχη, ρυθμίσεις ομάδας).
\item Ο χρήστης επιλέγει ένα άτομο απο τη λίστα μελών και εμφανίζονται πληροφορίες για το μέλος (όπως στην λίστα φίλων).
\item Ο χρήστης επιστρέφει στην προηγούμενη οθόνη.
\end{enumerate}

Εναλλακτική ροή 1:
\begin{enumerate}[label=3.\alph*.,ref=3.\alph*]
\item Ο χρήστης προσκαλεί σε μάχη όλα τα μέλη.
\item Το σύστημα αποστέλει ειδοποίηση σε κάθε μέλος της ομάδας.
\item O χρήσης επιλέγει να περιμένει ή να ξεκινήσει τη μάχη.
\item Το σύστημα υπολογίζει την συνολική δύναμη των συμμετεχόντων και "δημιουργεί" αντίπαλο ίσης δύναμης.
\item Η μάχη ξεκινάει όταν όλοι οι συμμετέχοντες επιβεβαιώσουν ότι είναι παρόντες.
\end{enumerate}

Εναλλακτική ροή 2:
\begin{enumerate}[label=3.\alph*.,ref=3.\alph*]
\item Ο χρήστης εισέρχεται στις ρυθμίσεις της ομάδας.
\item Εμφανίζεται η οθόνη ρυθμίσεων και διάφορα χαρακτηριστικά της ομάδας (όνομα, επιλογή διαγραφής, αρχηγός ομάδας, ...).
\item Ο χρήστης τροποποιεί κάποιο πεδίο.
\item Το σύστημα ζητάει επιβεβαίωσει. Στη συνέχεια, εφαρμόζει την αλλαγή και ενημερώνονται τα στοιχεία της ομάδας.
\end{enumerate}

Εναλλακτική ροή 2.2:
\begin{enumerate}[label=3.3.\alph*.,ref=3.3.\alph*]
\item Ο χρήστης δεν έχει τα κατάλληλα δικαιώματα ώστε να τροποποιήσει τις ρυθμίσεις τις ομάδας.
\item Το σύστημα εμφανίζει τα αντίστοιχα πεδία ως μη τροποποιήσιμα.
\item Ο χρήστης προσπαθεί να τροποποιήσει κάποιο πεδίο.
\item Το σύστημα τον ενημερώνει πως δεν έχει τα κατάλληλα δικαιώματα για αυτή την κίνηση.
\end{enumerate}


\item Ο χρήστης επιλέγει έναν φίλο.
\item Το σύστημα δείχνει πληροφορίες για τον φίλο (όνομα ομάδας, ενεργός/ανενεργός, τρέχον επίπεδο...) και ενέργεις που μπορεί να κάνει ο χρήστης (κατάργηση φίλου, πρόσκληση στην ομάδα, προβολή ομάδας...).
\item Ο χρήστης επιστρέφει στην προηγούμενη οθόνη.


\newpage
\subsection{Λίστα φίλων}
\label{sec:friendslist}
\begin{enumerate}
\item Ο χρήστης εισέρχεται στην οθόνη της λίστας φίλων.
\item Το σύστημα ανακτά τους φίλους του χρήστη και τους εμφανίζει.
\item Ο χρήστης στέλνει αίτημα φιλίας σε κάποιον άλλο παίκτη με βάση το όνομά του.
\item Το σύστημα ελέγχει το όνομα ώστε να υπάρχει στην βάση δεδομένων και επιστρέφει ενημερωτικό μήνυμα.
\item Ο χρήστης εξέρχεται από τα μενού προσθήκης φίλου.
\item Το σύστημα εμφανίζει επιλέον ενημερωτικό μήνυμα εάν ο άλλος παίκτης δεχθεί το αίτημα.
\end{enumerate}

Εναλλακτική Ροή 1:
\begin{enumerate}[label=2.1.\alph*.,ref=2.1.\alph*]
\item Το σύστημα βλέπει πως η συσκευή δεν έχει πρόσβαση στο διαδίκτυο.
\item Το σύστημα ανακτά την πιο πρόσφατα cached λίστα φίλων και την εμφανίζει.
\item Συνεχίζεται η κανονική ροή
\end{enumerate}

Εναλλακτική Ροή 2:
\begin{enumerate}[label=3.1.\alph*.,ref=3.1.\alph*]
\item Ο χρήστης επιλέγει να λάβει υπερσύνδεσμο για την πρόσκληση φίλου.
\item Το σύστημα εφόσον είναι συνδεδεμένο στο διαδίκτυο παράγει και εμφανίζει τον υπερσύνδεσμο που αρμόζει στον χρήστη.
\item Ο χρήστης εξέρχεται από τα μενού προσθήκης φίλου.
\item Το σύστημα εμφανίζει επιλέον ενημερωτικό μήνυμα εάν ο άλλος παίκτης δεχθεί το αίτημα.
\end{enumerate}

Εναλλακτική Ροή 3:
\begin{enumerate}[label=4.1.\alph*.,ref=4.1.\alph*]
\item Το σύστημα βρίσκει πως ο παίκτης με το εισαχθέντο όνομα δεν υπάρχει στην Β.Δ.
\item Ο χρήστης επαναλαμβάνει την εισαγωγή ονόματος, επανέρχοντας στην κανονική ροή στο βήμα 4.
\end{enumerate}

Εναλλακτική Ροή 4:
\begin{enumerate}[label=4.2.\alph*.,ref=4.2.\alph*]
\item Το σύστημα βλέπει πως η συσκευή δεν έχει πρόσβαση στο διαδίκτυο.
\item Το σύστημα ακυρώνει την διαδικασία προσθήκης φίλου, και ενημερώνει τον χρήστη με σχετικό μήνυμα για την απώλεια σύνδεσης στο διαδίκτυο. 
\end{enumerate}

\newpage
\subsection{Διαγραφή φίλων}
\label{sec:deletefriend}
\begin{enumerate}
\item Ο χρήστης εισέρχεται στην οθόνη λίστας φίλων.
\item Το σύστημα ανακτά τους φίλους του χρήστη και τους εμφανίζει.
\item Ο χρήστης επιλέγει να διαγράψει έναν φίλο του.
\item Το σύστημα ανανεώνει την βάση δεδομένων και την οθόνη του χρήστη.
\item Ο χρήστης εξέρχεται από τα μενού προσθήκης φίλου.
\end{enumerate}

Εναλλακτική Ροή 1:
\begin{enumerate}[label=2.\alph*.,ref=2.\alph*]
\item Το σύστημα βλέπει πως η συσκευή δεν έχει πρόσβαση στο διαδίκτυο.
\item Το σύστημα ανακτά την πιο πρόσφατα cached λίστα φίλων και την εμφανίζει.
\item Συνεχίζεται η κανονική ροή.
\end{enumerate}

\section{Λοιπές περιπτώσεις χρήσης}
Εδώ αναγράφονται οι περιπτώσεις χρήσεις που δεν έχουν βασική και εναλλακτική ροή. Παρατίθενται μόνο για την πληρότητα του τεχνικού κειμένου.
\subsection{Προβολή στατιστικών}
\subsection{Πώληση αντικειμένου}
\subsection{Υποβολή ticket υποστήριξης}
\subsection{Ανίχνευση επαναλήψεων}
\subsection{Επίπληξη παίκτη}
\subsection{Επεξεργασία σάκου παίκτη}
\subsection{Ανάληψη ticket υποστήριξης}
\subsection{Ανάκτηση ενεργών ticket υποστήριξης}
\subsection{Ρύθμιση λειτουργίας διακοσμιτή}