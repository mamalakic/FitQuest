\section*{Αλλαγές από προηγούμενη έκδοση}
%https://www.overleaf.com/learn/latex/Lists
Υπήρξαν πολλές αλλαγές λόγω της διαδικασίας ανάλυσης των περιπτώσεων χρήσης με διαγράμματα robustness και sequence. Στο κείμενο των περιπτώσεων χρήσης έχουμε σημειώσει με κόκκινο κάθε κομμάτι του κειμένου που είναι διαφορετικό. Ακολουθεί επεξήγηση για το τι αλλάχθηκε:
\subsection{Συνολικό μοντέλο περιπτώσεων χρήσης}
\begin{itemize}
\item Προστέθηκε η χρήση αναλώσιμου αντικειμένου σαν υπο-περίπτωση χρήστης στη μάχη.
\item Προστέθηκαν επιπλέον συνδέσεις include, exclude (χάρτης περιπέτειας προς ανίχνευση επαναλήψεων).
\end{itemize}
\subsection{Εξατομίκευση προγράμματος}
\begin{itemize}
    \item Η βασική ροή αλλάχθηκε απο την περίπτωση που ο χρήστης δεν έχει διαμορφώσει το προφίλ στην περίπτωση που το έχει διαμορφώσει, καθώς είναι πιο πιθανή περίπτωση.
    \item Αναλύθηκε περισσότερο η περίπτωση που ο χρήστης επιλέγει να ακολουθήσει πρόγραμμα που ακολούθησε στο παρελθόν.
    \item Προστέθηκε ο έλεγχος του αν ο χρήστης έχει αθληθεί πολλές φορές σε μικρό χρονικό διάστημα
    \item Πολλές λεκτικές αλλαγές και προσθήκες ώστε να είναι πιο ξεκάθαρη η φυσική γλώσσα.
\end{itemize}
\subsection{Αγορά αντικειμένου}
\begin{itemize}
    \item Ανάλυση διεπαφών χρήστη.
    \item Μετακίνηση εφαρμογής φίλτρων σε εναλλακτική ροή και ανάπτυξη κειμένου.
    \item Αναφορά σε βάση δεδομένων.
    \item Αλλαγή αναφοράς σε γενικό λάθος, σε συγκεκριμένα λάθος ελλειπούς νομισματικού υπολοίπου.
\end{itemize}
\subsection{Μάχη ομάδας}
\begin{itemize}
    \item Αναφορά σε βάση δεδομένων
    \item Αλλαγές ώστε να υποστηρίζεται η ασύγχρονη μάχη.
    \item Ανάλυση διεπαφών χρήστη.
    \item Διαγραφή ροής αποτυχίας μάχης.
    \item Προσθήκη ροής ψήφου σε περίπτωση μη ενεργής μάχης ομάδας.
\end{itemize}
\subsection{Σύστημα μάχης}
\begin{itemize}
    \item Εναλλακτική ροή 1.1 σε περίπτωση ανενεργότητας του παίκτη.
	\item Η ανίχνευση επαναλήψεων έγινε άλλο use case.
	\item Αντί για κάμερα, ο χρήστης μπορεί να πατάει κουμπί.
	\item Καταγραφή αρχείου μάχης.
	\item Διαγράφηκε μια εναλλακτική ροή που είχε μπει καταλάθως στο τεχνικό κείμενο.
\end{itemize}
\subsection{Χάρτης περιπέτειας}
\subsection{Σακίδιο αντικειμένων}
\subsection{Δημιουργία ομάδας}
\begin{itemize}
    \item Προστέθηκε έλεγχος για το αν ο χρήστης είναι σε ομάδα.
    \item Προστέθηκε εναλλακτική ροή που αναλύει την περίπτωση που τα πεδία δημιουργίας ομάδας δεν είναι έγκυρα.
    \item Πολλές λεκτικές αλλαγές και προσθήκες ώστε να είναι πιο ξεκάθαρη η φυσική γλώσσα.
\end{itemize}
\subsection{Λίστα φίλων}


\clearpage